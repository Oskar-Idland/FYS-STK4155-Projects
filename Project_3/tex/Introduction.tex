While artificial intelligence has revolutionized many fields, its impact on medical diagnostics has been particularly transformative through automated image analysis systems. The healthcare sector has increasingly adopted Convolutional Neural Networks (CNNs) for critical tasks like tumor detection and disease classification, where both accuracy and reliability are paramount.\cite{SHARMA2024637} However, implementing these systems in clinical settings presents unique challenges - traditional neural networks often require substantial computational resources and extensive training data, making them difficult to deploy in resource-limited environments\cite{howard2017mobilenetsefficientconvolutionalneural}. This has led to the development of various architectural approaches, from lightweight models trained from scratch to deep networks that leverage pre-trained weights from large datasets, each offering different tradeoffs between performance and resource requirements.

To address this, more efficient architectures like MobileNet and ResNet have been created. MobileNet is designed to be lightweight and efficient, making it ideal for tasks like medical image classification. ResNet, on the other hand, uses residual connections to help train deep networks, which solves the degradation problem and improves performance in complex image recognition tasks \cite{he2015deepresiduallearningimage}. Together, these architectures offer a range of solutions for efficient and effective deep learning in medical imaging applications.

This study aims to evaluate and compare the performance of various neural network architectures on a breast cancer histopathological dataset, focusing on three specific models: a custom CNN, MobileNet, and ResNet101. These models are selected based on their differing complexities, training requirements, and computational efficiencies. In this analysis, we will compare how well these models handle a small and imbalanced dataset, aiming to understand how different architectures affect classification accuracy and performance in medical applications.

This report is organized as follows: First, we introduce the theoretical concepts behind neural networks, CNNs, MobileNet, and ResNet, along with the relevant evaluation metrics. Next, we discuss the methods and implementation, including data preprocessing and model architectures. We then present the results, comparing the performance of the models. Finally, we conclude by summarizing our findings and identifying the most suitable model for handling complex data.