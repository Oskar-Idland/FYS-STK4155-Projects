
The ever-growing complexity of modern data calls for robust machine learning techniques capable of handling non-linear patterns and high-dimensional spaces. While traditional methods, such as Ordinary Least Squares (OLS), Ridge, and Lasso regression, offer valuable insights into model behavior and regularization, their limitations become apparent when addressing more intricate datasets. In our previous project, we explored these regression techniques in-depth, assessing their effectiveness on synthetic and real-world datasets, with particular attention to the bias-variance tradeoff and resampling methods for model validation\cite{Project1}. However, as data complexity increases, more sophisticated methods, such as neural networks, become necessary to model non-linear relationships and provide superior generalization.

In this project, we extend the concepts introduced earlier by diving into two fundamental areas of machine learning: optimization techniques and neural networks. These topics form the backbone of many modern machine learning algorithms and offer powerful tools for tackling a wide range of problems, from regression and classification to complex tasks in pattern recognition and forecasting. We implement and analyze these methods using both synthetic data (a simple 2 dimensional terrain data generated through the Franke function) and real-world data (the Wisconsin Breast Cancer dataset\cite{breast_cancer} for classification tasks).

Gradient descent plays a critical role in optimizing the parameters of complex models, especially when closed-form solutions are impractical or non-existent. In implementing Stochastic Gradient Descent (SGD), we explore various techniques for improving its convergence and stability. These include the incorporation of momentum and mini-batch processing, as well as adaptive learning rate methods like Adagrad, RMSprop, and Adam that automatically adjust the learning rate during training. Understanding how these adaptations affect the optimization process—from convergence speed to the ability to escape local minima—is crucial for developing efficient and robust machine learning models.

Furthermore, we implement feed-forward neural networks from scratch, providing a deep understanding of their inner workings. These models, inspired by biological neurons, consist of multiple layers of interconnected nodes that enable the learning of intricate data structures through non-linear activation functions. We investigate the impact of different activation functions (Sigmoid, RELU, and Leaky RELU), network architectures, and hyperparameters on model performance. By comparing our implementations with established libraries like Scikit-learn, we gain insights into both the theoretical foundations and practical considerations of neural network development.

The methods explored in this project have profound implications for modern machine learning applications. Efficient optimization techniques like SGD and its variants have made it possible to train large-scale neural networks on massive datasets, enabling breakthroughs in fields ranging from computer vision to natural language processing. Understanding these fundamental building blocks is crucial for developing more advanced machine learning systems and addressing increasingly complex real-world challenges. In section II, we present the theoretical framework underlying gradient descent optimization and neural networks,,with particular attention to the mathematical foundations of backpropagation and various activation functions. Once again, the majority of information is sourced from the lecture notes by Morten Hjorth-Jensen \cite{notes}. Section III details our methodology, including the implementation of both basic SGD and its advanced variants, as well as our neural network architecture design choices. In section IV, we present comprehensive results comparing the performance of different optimization methods and network configurations across both regression and classification tasks. Finally, section V summarizes our findings and discusses their implications for future applications in machine learning.

Through this comprehensive exploration of optimization methods and neural networks, we aim to bridge the gap between traditional regression techniques and modern deep learning approaches. Our analysis spans both regression and classification tasks, allowing us to evaluate the strengths and limitations of each method across different problem domains. The project not only builds upon our previous work with linear models but also establishes a foundation for understanding more advanced machine learning architectures and their optimization challenges.